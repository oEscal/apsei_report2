\section{Conclusão}


\subsection{O que torna o protocolo do Signal superior?}

Tendo sido adaptado por grandes empresas, como mencionado acima, não se consideraria um protocolo pouco robusto mas, para realçar uma das suas grandes vantagens, é o facto de ser \textbf{\textit{open-Source}}, o que permite que uma grande comunidade de pessoas com gosto por \textit{\textbf{infosec}} possam ver o \textit{source-code} e detetar falhas sem ser por tentativa e erro através de uma interface (como mencionado anteriormente). Enquanto o \textit{Signal} é \textit{Open-Source} e sabemos que a comunicação é realmente \textbf{\textit{end-to-end encrypted}}, e caso duvidemos, podemos facilmente ver o \textit{source-code} e confirmar. 

As outras ferramentas do mercado apresentam-se fechadas ao público e não permitem acesso ao código desta forma, apenas podemos confiar que as entidades se estão a comportar e não estão a fazer nada de errado com os nossos dados. Para além disso, como falado em cima, o \textit{Signal} tem sofrido bastantes auditorias e estas não encontraram qualquer problema com a aplicação.

Ao contrario do \textit{WhatApp}, o \textit{Signal} não guarda dados (ou metadados, que por vezes ainda são mais importantes) do cliente, e se este quiser privacidade, o \textit{Signal} garante-a, enquanto o \textit{WhatsApp}, sendo a aplicação mais utilizada, não o faz, o que põe em causa a privacidade de milhões de usuários. O \textit{WhatsApp}, apesar de usar o protocolo seguro e open-source do \textit{Signal} pode,\textit{behind the curtains}, estar a fazer algo mais que desconheçamos. O \textit{WhatsApp} efetua um \textit{load} de todos os nossos contactos do telemóvel para os seus servers remotos de forma a detetar quem possui ou não a aplicação, enquanto que o \textit{Signal} usa a tecnologia \textit{\textbf{SGX}} da \textit{Intel} \footnote{regiões protegidas da memória}. 

Comparativamente ao \textit{Telegram}, para além da distinção que o \textit{Telegram} não é \textit{open-source}, o \textit{Signal} perante as análises feitas, possui um algoritmo de segurança muito superior ao do \textit{Telegram}. O \textit{Signal} apaga \textbf{completamente} as mensagens passado algum tempo, enquanto que noutras \textit{apps}, \textit{Telegram} inclusive, os \textit{records} mantêm-se armazenados num servidor caso sejam apagadas.

Além de tudo o referido anteriormente, o \textit{Signal} encontra-se ativamente a procurar manter a privacidade dos seus clientes mesmo que sejam impostas regras pelos países onde estes vivem.
