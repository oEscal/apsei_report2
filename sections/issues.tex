\section{Problemas Sociais, Económicos e Éticos}

\subsection{Problemas Sociais}
Antes de poder haver qualquer discussão dos impactos sociais que uma aplicação deste género apresenta, é necessário perceber qual o publico alvo desta. Neste caso, o publico alvo do \textit{Signal} é qualquer pessoa que pretende realizar comunicações seguras e secretas \textit{online}. Tendo isso em consideração, temos pessoas como "maníacos" da cibersegurança, politicos, terroristas, foras da lei e qualquer um que pretenda fazer algo de forma mais escondida dos olhares mais atentos.

\subsubsection{Utilização do serviço para atividades ilegais}
Um dos problemas sociais derivados da utilização do \textit{Signal} que ocorre logo é a sua utilização para atividades ilegais. Recente à data de criação deste documento, é levado ao congresso Americano a proposta duma lei, \textit{Eliminating Abusive and Rampant Neglect of Interactive Technologies} ou \textit{EARN IT}, que prevê que qualquer \textit{website} combata ativamente abusos a menores feitos pelos seus utilizadores usando as \textit{features} do mesmo \cite{senate_earn_it}. Nela, é prevista a criação de "boas práticas", que minimizem ou ponham um termo no abuso de menores na \textit{internet}, por entidades superiores, sendo que serviços \textit{online} que prestem serviço nos Estados Unidos são obrigados a seguirem esse conjuntos de práticas. Apesar desta lei não indicar especificamente a não utilização de mecanismos de segurança que permitam utilizadores comunicar anonimamente, torna-se bastante claro que, se a ela for aceite, irão certamente haver "boas práticas" que irão passar pela existência de \textit{backdoors} em serviços que usem encriptação para proteger as comunicações, tal como referido por Riana Pfefferkorn, \textit{Associate Director of Surveillance and Cybersecurity} do Centro para Internet e Sociedade de Stanford, no \textit{blog post} \cite{stanford_cis}.

Claramente o serviço apresentado neste relatório seria um dos alvos mais óbvios das consequências desta lei, já que ao possuir \textit{E2E}, permite aos abusadores de menores ter um meio facilitado para praticarem este tipo de atos sem nenhuma autoridade desconfiar, nem mesmo a empresa por detrás do \textit{Signal}. Apesar deste crime ser um problema social sério, em que abusadores devem ser localizados e devidamente punidos, é necessário pensar noutros possíveis problemas que este tipo de censuras criam. Um deles é obviamente o perigo que uma lei desta dimensão causa na segurança das comunicações \textit{online},na liberdade de expressão. A primeira, porque existindo um \textit{backdoor}, é uma falha extrema de segurança e não se pode considerar uma comunicação segura uma comunicação onde seja minimamente possível por alguém obter o seu conteúdo e a segunda porque é mais um passo para o estado saber e controlar a forma como os cidadão do seu país interagem \textit{online}, algo que é obviamente muito perigoso e que dá mais poder ao estado do que aquele que ele necessita.

Para além disso, é necessário entender que o impedimento de encriptação neste tipo de redes sociais e a existência de \textit{backdoors} acessíveis às entidades judiciais só torna o trabalho destes criminosos mais difícil, mas não impossível, já que estes podem usar outros métodos menos convencionais para fazer as suas comunicações sem qualquer interferência das autoridades. Ou seja, este tipo de medidas só afeta a segurança das pessoas que comunicam na \textit{internet} e não afetaria de forma severa estes delinquentes \cite{sigan_blog_earn_it}.

A conclusão de tudo isto é que o \textit{Signal} ameaça deixar de prestar funções em solo Americano se a lei for passada, pelo que, como concluído no parágrafo anterior, só prejudica a segurança dos internautas.

\subsubsection{Utilização de domain fronting}
Em países, como o Irão ou Egito, onde é feita censura do que os seus cidadãos podem pesquisar na \textit{internet}, serviços como o \textit{Signal} ou o \textit{WhatsApp} são usualmente bloqueados por uma \textit{Firewall} do governo. Claramente estes regimes aplicam este tipo de medidas de forma a moldar fácilmente as ideias e instrução da sua sociedade, de forma a obter um melhor controlo sobre as mesmas e reduzir ou eliminar completamente possíveis protestos ou tentativas de colapso contra regimes autocráticos (como o Egito) e ditatoriais (como o Irão). 

Apesar dos esforços do \textit{Signal} se manter ativo nestes países através da técnica de \textit{domain fronting}, em 2018 a \textit{Google} e a \textit{Amazon} impossibilitaram a utilização deste método usando os seus serviços, tornando inacessível o uso do \textit{Signal} e outras aplicações que garantem comunicações seguras nos países descritos \cite{signal_amazon_letter}. É claro que quem o queira continuar a fazer, o pode consumar aproveitando-se, por exemplo, duma \textit{VPN}. Contudo, este tipo de ferramentas são mais disseminadas por pessoas informadas do ramo da informática, pelo que para um "cidadão comum" pode não ser o mais evidente.

Apesar dos esforços que este género de registes tem feito para limitar a disseminação de informação nos seus países, têm felizmente continuado a haver esforços de serviços como o \textit{Signal} para continuar a haver fontes de divulgação de conhecimento onde este é escaço.


\subsection{Problemas económicos}
Como já explicado no inicio deste relatório, o \textit{Signal} pertence a uma fundação sem fins lucrativos, pelo que não possui qualquer modelo de negócio, ou pelo menos um claro. A sua principal ambição é criar o ambiente mais seguro possível para as comunicações realizadas pelos seus utilizadores. Por esse motivo, os problemas económicos que ele pode criar advém mais do mercado que tiram à sua concorrência do que do próprio modelo de negócio.

Sendo o seu principal oponente o \textit{WhatsApp}, pode-se facilmente entender que, apesar do \textit{Signal} ter sido lançado para o publico em 2014, desde então não parece ter tirado "grande terreno" ao seu rival, sendo que este possui aproximadamente mil milhões e meio de utilizadores ativos mensalmente e o \textit{Signal} possuir mais de 10 milhões de \textit{downloads} no \textit{Google Play} (não existe quaisquer dados públicos das estatísticas da utilização do serviço, pelo que se torna fazer uma comparação de outra forma). Mesmo em relação ao \textit{Telegram}, que possui cerca de 200 milhões de utilizadores mensais ativos, outra aplicação que oferece \textit{E2E}, é fácil entender que apresenta uma muito menor quantidade de utilizares \cite{other_apps_statistics}. 

Conclui-se assim, que duma forma geral, este serviço não tem uma presença acentuada no mercado, pelo que não pode apresentar problemas económicos graves no grande panorâma.

Contudo, pelo facto de não apresentar um modelo económico usual, isso pode traduzir-se em problemas económicos internos. Pelo facto do serviço pertencer a uma associação sem fins lucrativos, torna-se impraticável gerar dinheiro da forma típica, sobrevivendo apenas através de doações. Para além de isso implicar a existência duma equipa de programadores bastante reduzida (falamos aqui dos trabalhadores pagos, não da comunidade \textit{open source}), é também impossível, por exemplo, financiar ações judiciais que possam haver contra a aplicação. Por esse motivo é que à data da criação deste documento o \textit{Signal} anuncia uma possível cessão de funções em solo americano caso a lei \textit{EARN IT} seja aprovada, já que não possui quaisquer meios económicos que suportem possíveis ações judiciais dos seus utilizadores, ao contrário das empresas concorrentes, que possuem um modelo financeiro bem estruturado.


\subsection{Problemas éticos}
Como o \textit{Signal} é um meio seguro para fazer comunicações, os seus problemas éticos vão muito de encontro aos problemas éticos do uso de encriptação no geral. Como discutido nos problemas sociais, este tipo de serviços são normalmente usados para praticar crimes, como abuso de menores ou planeamento de ataques terroristas. Contudo, é também através deles que muitos cidadãos possuem uma ferramenta para disseminar informação, sendo por isso um meio importante para a existência de pensamento critico livre em países onde não o é permitido. Será que o bom compensa o mau no uso deste aplicação? Claramente, não existe uma resposta correta, mas podemos fazer várias constatações. Como já falado antes, quem quer cometer este tipo de crimes, não necessita do \textit{Signal} para o fazer, há métodos menos convencionais fora do controlo do governo que funcionam da mesma maneira. Contudo, o que se pode dizer no caso de pessoas que vivem num regime de opressão? Pessoas que nunca tiveram o acesso que nós, num país democrático, temos desde sempre à tecnologia, e que por isso não têm conhecimentos suficientes para saber sequer da existência de outros meios para fazer comunicações seguras e anónimas que não uma simples aplicação de telemóvel? 

Não podemos dar uma resposta com toda a certeza de se é ou não moralmente correta a existência duma ferramenta deste tipo, mas podemos dar a nossa opinião dados os argumentos anteriores. Para nós, apesar do mal que advém da existência dela, a privacidade é um direito importante que, enquanto pessoas, temos e que não deve ser reduzido ou posto de parte por haver quem o use para atividades maliciosas.
