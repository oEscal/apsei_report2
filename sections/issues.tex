\section{Problemas Sociais, Económicos e Éticos}

\subsection{Problemas Sociais}
Antes de poder haver qualquer discussão dos impactos sociais que uma aplicação deste género apresenta, é necessário perceber qual o publico alvo desta. Neste caso, o publico alvo do \textit{Signal} é qualquer pessoa que pretende realizar comunicações seguras e secretas \textit{online}. Tendo isso em consideração, temos pessoas como "maníacos" da cibersegurança, politicos, terroristas, foras da lei e qualquer um que pretenda fazer algo de forma mais escondida dos olhares mais atentos.

\subsubsection{Utilização do serviço para atividades ilegais}
Um dos problemas sociais derivados da utilização do \textit{Signal} que ocorre logo é a sua utilização para atividades ilegais. Recente à data de criação deste documento, é levado ao congresso Americano a proposta duma lei, \textit{Eliminating Abusive and Rampant Neglect of Interactive Technologies} ou \textit{EARN IT}, que prevê que qualquer \textit{website} combata ativamente abusos a menores feitos pelos seus utilizadores usando as \textit{features} do mesmo \cite{senate_earn_it}. Nela, é prevista a criação de "boas práticas", que minimizem ou ponham um termo no abuso de menores na \textit{internet}, por entidades superiores, sendo que serviços \textit{online} que prestem serviço nos Estados Unidos são obrigados a seguirem esse conjuntos de práticas. Apesar desta lei não indicar especificamente a não utilização de mecanismos de segurança que permitam utilizadores comunicar anonimamente, torna-se bastante claro que, se a ela for aceite, irão certamente haver "boas práticas" que irão passar pela existência de \textit{backdoors} em serviços que usem encriptação para proteger as comunicações, tal como referido por Riana Pfefferkorn, \textit{Associate Director of Surveillance and Cybersecurity} do Centro para Internet e Sociedade de Stanford, no \textit{blog post} \cite{stanford_cis}.

Claramente o serviço apresentado neste relatório seria um dos alvos mais óbvios das consequências desta lei, já que ao possuir \textit{E2E}, permite aos abusadores de menores ter um meio facilitado para praticarem este tipo de atos sem nenhuma autoridade desconfiar, nem mesmo a empresa por detrás do \textit{Signal}. Apesar deste crime ser um problema social sério, em que abusadores devem ser localizados e devidamente punidos, é necessário pensar noutros possíveis problemas que este tipo de censuras criam. Um deles é obviamente o perigo que uma lei desta dimensão causa na segurança das comunicações \textit{online},na liberdade de expressão. A primeira, porque existindo um \textit{backdoor}, é uma falha extrema de segurança e não se pode considerar uma comunicação segura uma comunicação onde seja minimamente possível por alguém obter o seu conteúdo e a segunda porque é mais um passo para o estado saber e controlar a forma como os cidadão do seu país interagem \textit{online}, algo que é obviamente muito perigoso e que dá mais poder ao estado do que aquele que ele necessita.

Para além disso, é necessário entender que o impedimento de encriptação neste tipo de redes sociais e a existência de \textit{backdoors} acessíveis às entidades judiciais só torna o trabalho destes criminosos mais difícil, mas não impossível, já que estes podem usar outros métodos menos convencionais para fazer as suas comunicações sem qualquer interferência das autoridades. Ou seja, este tipo de medidas só afeta a segurança das pessoas que comunicam na \textit{internet} e não afetaria de forma severa estes delinquentes \cite{sigan_blog_earn_it}.

A conclusão de tudo isto é que o \textit{Signal} ameaça deixar de prestar funções em solo Americano se a lei for passada, pelo que, como concluído no parágrafo anterior, só prejudica a segurança dos internautas.

\subsubsection{Utilização de domain fronting}
Em países, como o Irão ou Egito, onde é feita censura do que os seus cidadãos podem pesquisar na \textit{internet}, serviços como o \textit{Signal} ou o \textit{WhatsApp} são usualmente bloqueados por uma \textit{Firewall} do governo. Claramente estes regimes aplicam este tipo de medidas de forma a moldar fácilmente as ideias e instrução da sua sociedade, de forma a obter um melhor controlo sobre as mesmas e reduzir ou eliminar completamente possíveis protestos ou tentativas de colapso contra regimes autocráticos (como o Egito) e ditatoriais (como o Irão). 

Apesar dos esforços do \textit{Signal} se manter ativo nestes países através da técnica de \textit{domain fronting}, 

\begin{itemize}
   \item EARN IT Act (child abuse law to implement backdoors on social messaging apps like signal)
   \begin{itemize}
      \item \url{https://www.pcmag.com/news/messaging-app-signal-threatens-to-dump-us-market-if-anti-encryption-bill}
      \item por um lado, abusos de menores é um problema sério, que tem de ser reduzido ou extreminado de alguma meneira
      \item contudo, não é com este tipo de leis, já que abusadores podem sempre usar outras aplicações mais "não convencionais" 
      \item outro problema é o facto deste tipo de leis reduzirem a liberdade da população duma sociedade: "But other lawmakers say they're against the bill, citing its potential to be abused. "This terrible legislation is a Trojan horse to give Attorney General Barr and Donald Trump the power to control online speech and require government access to every aspect of Americans' lives," said Senator Ron Wyden (D-Oregon) last month."
      \item esta lei obviamente traz coimas, coisa que empresas grandes como o facebook ou google conseguem suportar, mas que o signal não consegue, porque é apenas um projeto duma associação sem fins lucrativos
      \item \url{https://signal.org/blog/earn-it/}
      \item vai tudo contra liberdade de expressão
      \item contra a lei "Section 230 of the Communications Decency Act", que dita que qualquer aplicação ou website não é responsável pelo conteudo que os seus utilizadores publicam na internet
   \end{itemize}
   \item não é uma aplicação intrusiva -> não tem anúncios
   \item não possui um marketing forte como outros rivais de grandes empresas, pelo que sofre pela impopularidade
   \item problema de mensagens com limite de tempo (auto destroem-se)
   \item utilização de domain fronting \url{https://signal.org/blog/looking-back-on-the-front/}, o que torna possivel usar uma aplicação com este tipo de poder num país onde são banidas (por exemplo, no Irão, o WhatsApp está banido)
   \item o protocolo tem sido usado por outras aplicações de concorrência, instigando a comunicação segura na internet -> pontos fortes e fracos
   \item problemas com a implementação de backdoors também na Australia \url{https://qz.com/1497092/the-signal-encrypted-app-service-wont-comply-with-australias-assistance-and-access-bill/}
\end{itemize}

