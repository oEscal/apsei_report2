\section{Ferramenta}

\vspace{-0.1cm}
\subsection{O que é o Signal?}
\label{sec:signal}

É um serviço de mensagens encriptadas multi-plataformas, como o \textit{WhatsApp} ou o \textit{Facebook Messenger}, desenvolvida pelo \textit{Signal Foundation} e \textit{Signal Messenger}, em que o seu foco principal é a privacidade e segurança dos utilizadores.
Esta aplicação é grátis e está disponível para  \textit{Android}, \textit{iOS} e aplicação \textit{Desktop},
que para além dos protocolos extra de segurança, incluem também as ferramentas básicas características dos serviços de mensagens, como chamadas individuais e de grupo, chamadas de vídeo e de voz, suporte de \textit{emojis}, \textit{stickers}, entre outros.

O \textit{Signal} usa o número de telemóvel para identificação do utilizador, evitando o processo de memorização de credenciais, e usa encriptação \textit{end-to-end} para proteger todas as comunicações entre clientes da aplicação.

Além destas características, este serviço inclui mecanismos de verificação de identidade dos contactos e integridade dos dados de um canal.

Este serviço é considerado superior por aqueles que se preocupam com segurança e privacidade dos dados, já que todos os dados enviados ou recebidos são encriptados, o que dificulta bastante a compreensão dos dados caso algum sujeito consiga interceta-los (\textit{man-in-the-middle}).
Para além destas características, o \textit{Signal} não armazena qualquer tipo de dados (\textit{meta dados}) dos utilizadores, o que impossibilita outras entidades externas de lhes terem acesso para outros fins (por exemplo, o governo em processos de investigação não pode pedir acesso aos dados, porque estes simplesmente não existem, são como que uma \textit{black box} encriptada). Outro processo que costuma ocorrer com outras aplicações é a própria vender estes dados a outros serviços para fins publicitários, o que, mais uma vez, é impossível com o \textit{Signal}. Em suma, como a aplicação não armazena este tipo de dados, estes não podem ser vazados, o que indica que todo o tipo de dados pessoais permanecem privados e seguros.

Para além disso, todo o código desenvolvido para a realização desta aplicação é \textit{open source} (\hyperlink{https://github.com/signalapp}{\textit{Signal} repository}), o que possibilita qualquer sujeito observar o código fonte da aplicação. Isto possibilita que os utilizadores possam comprovar que o \textit{Signal} mantém os altos padrões de privacidade, que a própria aplicação afirma ter. Acoplado a estas características, este serviço é livre de publicidade \cite{signal_wikipedia, signal_popular}.

\subsection{Comparações com aplicações semelhantes}
\label{sec:comparacoes}
Para clarificar ainda mais o porquê desta aplicação ser superior às da concorrência, será feita uma comparação com os principais serviços de mensagens da atualidade.

Como se pode verificar na tabela \ref{tab:comparacoes}, o \textit{Signal} é a aplicação que oferece a maior taxa de segurança e privacidade, pois oferece encriptação de todos os dados (incluindo os meta dados) e não partilha os dados dos utilizadores com entidades externas para fins publicitários.
Outro aspeto a salientar é o facto de as duas aplicações mais usadas globalmente serem também as duas piores aplicações em termos de privacidade e segurança dos dados, pois estas armazenam e guardam os dados (e meta dados) dos utilizadores com objetivo de os venderem futuramente a entidades externas com objetivos publicitários, aproveitando-se do objetivo da aplicação para ter fontes de rendimentos externas.

\begin{table}[h]

\centering
\begin{adjustbox}{width=1\textwidth}
\begin{tabular}{lllllll}
                                & Facebook Messenger & WhatsApp & iMessage & Telegram & Wire & Signal \\
Não armazena dados dos clientes & Não                & Não      & Não      & Não      & Sim  & Sim    \\
Encriptação por padrão          & Não                & Sim      & Sim      & Não      & Sim  & Sim    \\
Open Source                     & Não                & Não      & Não      & Não      & Sim  & Sim    \\
Meta dados encriptados          & Não                & Não      & Não      & Não      & Não  & Sim    \\
Recusa a partilha dos dados dos utilizadores com agências publicitárias & Não & Não & Sim & Sim & Sim & Sim
\end{tabular}
\label{tab:comparacoes}
\end{adjustbox}
\caption{Comparações entre as aplicações mais usadas no mercado \cite{apps_comparation}.}
\end{table}


\subsection{Dados estatísticos}
\label{sec:estatis}
Para o objetivo deste trabalho, é um excelente complemento a amostragem de estatísticas e gráficos para simplificar a explicação e visualização de dados sobre a utilização da plataforma, de forma a uma melhor compreensão do tópico abordado.
Após uma pesquisa sobre tais dados, não nos foi possível encontrar quaisquer tipo de informação deste tipo, despoletando uma segunda pesquisa que justificasse a não existência de tais tipos de dados. Chegamos por fim à conclusão de que estes não existem, já que o objetivo principal deste serviço é a privacidade dos seus utilizadores.


\subsection{Recomendações}
\label{sec:recomendacoes}
Houveram também algumas recomendações desta aplicação por parte de entidades mundialmente conhecidas e que possuem renome na área da segurança:
\begin{itemize}
    \item \textbf{\textit{Edward Snowden}}: Analista de sistemas, ex-administrador de sistemas da \textit{CIA}, ex-contratado da \textit{NSA}. Realizou um \textit{post} na rede social \textit{Twitter} a recomendar a aplicação \textit{Signal}: "I use Signal every day. \#notesforFBI (Spoiler: they already know)"  \cite{Snowden_tweet}.
    \item \textbf{Comissão europeia}: recomenda \textit{Signal} em vez de \textit{WhatsApp}, pedindo aos seus funcionários para usarem esta aplicação de mensagens, em detrimento de outras, por a considerarem mais segura \cite{publico_new}.
\end{itemize}
