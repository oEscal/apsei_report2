\section{Ferramenta}

\vspace{-0.1cm}
\subsection{O que é o Signal?}
\label{sec:signal}

É uma serviço de mensagens encriptadas multi-plataformas,como o \textit{WhatsApp ou Facebook Messenger}, desenvolvida por \textit{Signal Foundation} e \textit{Signal Messenger}, em que o seu foco principal é a privacidade e segurança dos utilizadores.
Esta aplicação é grátis e esta disponível para  Android, iOS e Chrome (através de uma extensão),
para além dos protocolos extras de segurança, inclui também as ferramentas básicas características dos serviços de mensagens, incluindo chamadas individuais e de grupo, chamadas de vídeo e de voz, suporte de \textit{emojis}, etc.\newline
\textit{Signal} usa o numero de telemóvel para identificação do utilizador, evitando o processo de memorização de credenciais e usa encriptação \textit{end-to-end} para proteger todas as comunicações entre clientes da aplicação.\newline
Além destas características, este serviço inclui mecanismos de verificação de identidade dos contactos e integridade dos dados de um canal.\newline
Este serviço é considerado superior por aqueles que se preocupam principalmente com segurança e privacidade dos dados, todos os dados enviados ou recebidos são encriptados, o que dificulta bastante a compreensão dos dados caso algum sujeito consiga intercepta-los.
Para além destas características, o Signal não armazena qualquer tipo de dados (\textit{meta dados}) dos utilizadores, o que impossibilita outras entidades externas terem acesso a estes dados para outros fins, por exemplo, o governo em processos de investigação não pode pedir acesso aos dados porque estes simplesmente não existem. Outro processo que costumam ocorrer com outras aplicações é a própria aplicação vender estes dados a outros serviços para fins publicitários, o que não acontece no \textit{Signal}. Em suma, como a aplicação não guarda este tipo de dados, estes não podem ser vazados, o que indica que todo o tipo de dados pessoais permanecem privados e seguros.\newline
Todo o código desenvolvido para a realização desta aplicação é \textit{open source} (\hyperlink{https://github.com/signalapp}{\textit{Signal} repository}), o que possibilita qualquer sujeito observar o código fonte da aplicação, significando que os utilizadores podem comprovar que os trabalhadores mantêm os altos padrões de privacidade, que a própria empresa afirma ter. Acoplado a estas características, este serviço é livre de publicidades. \cite{signal_wikipedia} \cite{signal_popular}

\subsection{Comparações com aplicações semelhantes}
\label{sec:comparacoes}
Para clarificar ainda mais o porquê desta aplicação ser superior às aplicações da concorrência, será feito uma comparação com os principais serviços de mensagens actualmente.

Como se pode verificar na tabela \ref{tab:comparacoes}, \textit{Signal} é a aplicação que oferece a maior taxa de segurança e privacidade,pois esta oferece encriptação de todos os dados (incluindo os meta dados) e não partilha os dados dos utilizadores para entidades externas para fins publicitários.
Outro aspecto a salientar é o facto de as duas aplicações mais usadas globalmente serem também as duas piores aplicações em termos de privacidade e segurança dos dados, pois estas armazenam e guardam os dados (e meta dados) dos utilizadores com objectivo de os venderem futuramente para entidades externas com objectivos publicitários, aproveitando-se do objectivo da aplicação para ter fontes de rendimentos externas.

\begin{table}[!h]

\centering
\begin{adjustbox}{width=1\textwidth}
\begin{tabular}{lllllll}
                                & Facebook Messenger & WhatsApp & iMessage & Telegram & Wire & Signal \\
Não armazena dados dos clientes & Não                & Não      & Não      & Não      & Sim  & Sim    \\
Encriptação por padrão          & Não                & Sim      & Sim      & Não      & Sim  & Sim    \\
Open Source                     & Não                & Não      & Não      & Não      & Sim  & Sim    \\
Meta dados encriptados          & Não                & Não      & Não      & Não      & Não  & Sim    \\
Recusa a partilha dos dados dos utilizadores com agências publicitárias & Não & Não & Sim & Sim & Sim & Sim
\end{tabular}
\label{tab:comparacoes}
\end{adjustbox}
\caption{Comparações entre as aplicações mais usadas no mercado}
\cite{apps_comparation}
\end{table}


\subsection{Dados estatísticos}
\label{sec:estatis}
Para o objectivo deste trabalho é um excelente complemento a amostragem de dados estatísticos e gráficos para simplificar a explicação e visualização de dados para melhor compreensão do tópico abordado.
Após alguma pesquisa exaustiva sobre tais dados, não nos foi possível encontrar quaisquer tipo de informação deste tipo, despoletando uma segunda pesquisa que justificasse a não existência de tais tipos de dados. Chegando à conclusão que o próprio \textit{CEO} afirmou que este tipo de informação não esta disponível ao publico.


\subsection{Recomendações}
\label{sec:recomendacoes}
Existiram algumas recomendações desta aplicação por parte de entidades mundialmente conhecidas.
\begin{itemize}
    \item \textit{Edward Snowden}: Analista de sistemas ,ex-administrador de sistemas da CIA, ex-contratado da NSA. Realizou um \textit{post} na rede social \textit{Twitter} a recomendar a aplicação \textit{Signal}: "I use \textit{Signal} every day. \#notesforFBI (Spoiler: they already know)"  \cite{Snowden_tweet}
    \item Comissão europeia recomenda \textit{Signal} em vez de \textit{WhatsApp}, pedindo aos seus funcionários para usarem a aplicação de mensagens, em detrimento de outras, por considerarem a aplicação \textit{Signal} mais segura \cite{publico_new}
\end{itemize}
